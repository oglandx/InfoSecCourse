\documentclass[a4paper]{article}
\usepackage[utf8x]{inputenc}
\usepackage[T1,T2A]{fontenc}
\usepackage[russian]{babel}
\usepackage{hyperref}
\usepackage{indentfirst}
\usepackage{listings}
\usepackage{color}
\usepackage{here}
\usepackage{array}
\usepackage{multirow}
\usepackage{graphicx}
\usepackage[space]{grffile}

\usepackage{caption}
\renewcommand{\lstlistingname}{Программа} % заголовок листингов кода

\usepackage{listings}
\lstset{ %
extendedchars=\true,
keepspaces=true,
language=bash,					% choose the language of the code
basicstyle=\footnotesize,		% the size of the fonts that are used for the code
numbers=left,					% where to put the line-numbers
numberstyle=\footnotesize,		% the size of the fonts that are used for the line-numbers
stepnumber=1,					% the step between two line-numbers. If it is 1 each line will be numbered
numbersep=5pt,					% how far the line-numbers are from the code
backgroundcolor=\color{white},	% choose the background color. You must add \usepackage{color}
showspaces=false				% show spaces adding particular underscores
showstringspaces=false,			% underline spaces within strings
showtabs=false,					% show tabs within strings adding particular underscores
frame=single,           		% adds a frame around the code
tabsize=2,						% sets default tabsize to 2 spaces
captionpos=b,					% sets the caption-position to bottom
breaklines=true,				% sets automatic line breaking
breakatwhitespace=false,		% sets if automatic breaks should only happen at whitespace
escapeinside={\%*}{*)},			% if you want to add a comment within your code
postbreak=\raisebox{0ex}[0ex][0ex]{\ensuremath{\color{red}\hookrightarrow\space}}
}

\usepackage[left=2cm,right=2cm,
top=2cm,bottom=2cm,bindingoffset=0cm]{geometry}



\begin{document}	% начало документа

\begin{titlepage}	% начало титульной страницы

	\begin{center}		% выравнивание по центру

		\large Санкт-Петербургский Политехнический Университет Петра Великого\\
		\large Институт компьютерных наук и технологий \\
		\large Кафедра компьютерных систем и программных технологий\\[6cm]
		% название института, затем отступ 6см
		
		\huge Методы и средства защиты информации\\[0.5cm] % название работы, затем отступ 0,5см
		\large Отчет по лабораторной работе №4\\[0.1cm]
		\large Набор инструментов для аудита беспроводных сетей AirCrack\\[5cm]

	\end{center}


	\begin{flushright} % выравнивание по правому краю
		\begin{minipage}{0.25\textwidth} % врезка в половину ширины текста
			\begin{flushleft} % выровнять её содержимое по левому краю

				\large\textbf{Работу выполнил:}\\
				\large Косолапов С.А.\\
				\large {Группа:} 53501/3\\
				
				\large \textbf{Преподаватель:}\\
				\large Вылегжанина К.Д.

			\end{flushleft}
		\end{minipage}
	\end{flushright}
	
	\vfill % заполнить всё доступное ниже пространство

	\begin{center}
	\large Санкт-Петербург\\
	\large \the\year % вывести дату
	\end{center} % закончить выравнивание по центру

\thispagestyle{empty} % не нумеровать страницу
\end{titlepage} % конец титульной страницы

\vfill % заполнить всё доступное ниже пространство


% Содержание
\tableofcontents
\newpage


\section{Цель работы}

Изучить основные возможности пакета AirCrack и принципы взлома WPA/WPA2 PSK и WEP.

\section{Программа работы}

\subsection{Изучение}

\begin{enumerate}

\item Изучить документацию по основным утилитам пакета – airmon-ng, airodump-ng, aireplay-ng, aircrack-ng.

\item Запустить режим мониторинга на беспроводном интерфейсе

\item Запустить утилиту airodump, изучить формат вывода этой утилиты, форматы файлов, которые она может создавать

\end{enumerate}

\subsection{Практическое задание}

Проделать следующие действия по взлому WPA2 PSK сети (описание по ссылке "Руководство по взлому WPA"в материалах):

\begin{enumerate}

\item Запустить режим мониторинга на беспроводном интерфейсе

\item Запустить сбор трафика для получения аутентификационных сообщений

\item Если аутентификаций в сети не происходит в разумный промежуток времени, произвести деаутентификацию одного из клиентов, до тех пор, пока не удастся собрать необходимых для взлома аутентификационных сообщений

\item Произвести взлом используя словарь паролей

\end{enumerate}

\section{Изучение}

\subsection{Изучить документацию по основным утилитам пакета – airmon-ng, airodump-ng, aireplay-ng, aircrack-ng}

\begin{itemize}

\item \textbf{airmon-ng}

-- 	утилита для выставления различных карт в режим мониторинга.

\item \textbf{airodump-ng}

-- утилита, позволяющая захватывать пакеты протокола 802.11.

\item \textbf{aireplay-ng}

-- утилита для генерации трафика, необходимого для взлома утилитой aircrack-ng.

\item \textbf{aircrack-ng}

-- утилита для взлома ключей WPA и WEP с помощью перебора по словарю.

\end{itemize}

\subsection{Запустить режим мониторинга на беспроводном интерфейсе}

\lstinputlisting[numbers=none, keywords={}]{logs/airmon.txt}

\subsection{Запустить утилиту airodump, изучить формат вывода этой утилиты, форматы файлов, которые она может создавать}

\lstinputlisting[numbers=none, keywords={}]{logs/airodump-help.txt}

Сохранить дамп можно с помощью опции --write или -w.

Также нас интересует ключ output-format. Там указано, что формат выходного файла может быть: pcap, ivs, csv, gps, kismet, netxml.

Наиболее интересны pcap-файлы, потому что содержат всю перехваченную информацию. Открыть такие файлы можно, например, с помощью wireshark'а.

\section{Практическое задание}

\subsection{Запустить режим мониторинга на беспроводном интерфейсе}

\lstinputlisting[numbers=none, keywords={}]{logs/airodump-monitor.txt}

Так как мой роутер был безжалостно уничтожен во время подключения нового провайдера, подопытными в данном случае выступят мои соседи (кто эти люди?) с MAC-адресом E0:3F:49:8A:44:30 и отсутствующим в публичном доступе имени сети.


\subsection{Запустить сбор трафика для получения аутентификационных сообщений}

\lstinputlisting[numbers=none, keywords={}]{logs/airodump-beacon.txt}

\subsection{Если аутентификаций в сети не происходит в разумный промежуток времени, произвести деаутентификацию одного из клиентов, до тех пор, пока не удастся собрать необходимых для взлома аутентификационных сообщений}

\lstinputlisting[numbers=none, keywords={}]{logs/aireplay-deauth.txt}

Как сначала был выбран нужный канал, а затем проведена деаутентификация.

После этого можем писать дамп, пока не найдём handshake.

\lstinputlisting[numbers=none, keywords={}]{logs/handshake.txt}

\subsection{Произвести взлом используя словарь паролей}

\lstinputlisting[numbers=none, keywords={quake2016}]{logs/aircrack.txt}

Таким образом, у меня получилось подобрать пароль, используя словарь.

\section{Выводы}

Взломать беспроводную сеть с WPA/WPA2 PSK с помощью утилит, входящих в состав AirCrack, при условии, что есть словарь возможных комбинаций паролей, довольно просто. И главной проблемой взлома таких сетей является отсутствие словаря в произвольном случае. Перебор может занять "весьма существенное" время, и это не является хорошим вариантом для взлома защищённых сетей. С другой стороны, это в немалой мере гарантирует, что частная сеть не будет взломана злоумышленниками за малое время, при условии достаточно хорошо составленного или сгенерированного пароля, не входящего в словари.

Пакет AirCrack также позволяет прослушивать пакеты, создавать новые на их основе, производить деаутентификацию клиента сети и много других вещей, которые могут пригодиться при подготовке и проведении атаки.

\end{document}